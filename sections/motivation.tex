\section{Motivation}    \label{sec:motivation}
% Motivation, problem statement, related work

%% Problem statement
% Soft robotic systems require methods of actuation that do not inhibit their compliant structure. Parallel combinations of fiber-reinforced fluid-driven actuators are capable of generating controllable spacial forces without imposing rigidity.

%% What are soft robots, why are they useful?
Soft robotic systems have the potential to offer capabilities that go far beyond those of traditional rigid-bodied robots.
Their compliant structure allows them to adapt their overall shape to navigate unstructured environments, to safely work alongside humans, to manipulate delicate goods, and to absorb impacts without damage \cite{majidi2014soft}. 

%% Fluid-driven actuators are a good choice for soft robots
To obtain all of these advantages, soft robotic systems require actuators that can produce forces without imposing a rigid structure. As a result, a large number of soft robotic systems are actuated by fluid-driven soft actuators \cite{grissom2006design, hawkes2017soft, marchese2014autonomous, tolley2014resilient}. 
In these actuators, a pressurized fluid such as water or air induces a targeted deformation of a soft structure enclosing a fluid-filled cavity. 
To achieve a specific type and direction of deformation, and not merely a homogeneous expansion, the stiffness of the soft structure is pattered by adding reinforcing elements such as fibers, beams, or plates \cite{galloway2013mechanically, marchese2015recipe, rus2015design}.

%% Existing literature on the control of soft robots rely mainly on feed-forward control (with some notable exceptions). Goal of this paper is to demonstrate the capability of a model-based approach to achieve control
The most common approach to controlling systems that are actuated by fluid-driven soft actuators is model-free feed-forward, such as in soft crawling robots (cite), fish (cite), etc (include more examples and citations). Such an approach is sufficient to demonstrate the novel capabilities of a soft structure or to generate repeated motions, but falls short in more complex tasks where non-periodic position/force control is desired. \Dan{Should probably also mention model-free PID} 

%% Model-based control approach enables control of more complex behavior
In this work we propose and validate a model-based control approach founded on the notion of a \emph{fluid Jacobian}, which linearly maps the geometrical deformation of a soft actuator, or of a system of actuators, to a change in their volume. 
Due to its linear structure, this model enables the efficient numerical calculation of control inputs given a desired state.
%The linear structure of this model enables efficient calculation of control inputs given a desired state.
%% FREEs are customizable and well-suited for parallel combinations
We demonstrate the efficacy of this approach by controlling the position of an end effector connected to parallel configurations of fiber-reinforced soft actuators, known as Fiber-Reinforced Elastomeric Enclosures (FREEs). 




% FREEs are used due to their versatility, % customizability?  
% since by merely changing the arrangement of a FREE's fiber reinforcements, it can be customized to yield a desired motion or force profile without constraining motion to occur exclusively in the direction of force \cite{bishop2015design, connolly2015mechanical, felt2018closed, krishnan2015kinematics, bishop2013force, bruder2017model, sedal2017constitutive}. This property makes them well suited to be combined in parallel, where the forces of individual actuators are superimposed to generate a multidimensional spacial force (cite IROS). 
% \Dan{I want it to be clear that the approach we use is not specific to FREEs, we have just chosen them to demonstrate our model. But I don't know if introducing them at the end of the section like this is best.}





%% TEXT FROM IROS PAPER FOR REFERENCE %%%%%%%%%%%%%%%%%%%%%%%%%%%%%%%%%%%%%%%%%%%%%%%%%%%%%%%%%%%%%

% Examples of these actuators include bellows \cite{pridham1967bellows}, pneu-nets \cite{mosadegh2014pneumatic}, McKibben muscles \cite{tondu2012modelling}.

%% FREEs are great because they are fluid-driven and customizable
%A particularly promising type of soft fluid-driven actuator is the fiber-reinforced elasomeric enclosure (FREE) \cite{bishop2015design}. 
%A FREE consists of a fluid-filled elastomeric tube wound with reinforcing fibers that pattern its stiffness to yield a desired mode and direction of deformation upon pressurization. %(Fig.~\ref{fig:FREEhand}). 
%By changing the arrangement of these fibers, a FREE can be customized to yield a variety of desired deformations and forces \cite{bishop2015design}. 


%% Basic argument: Rigid robots: structure constrained, actuators exert forces at joints to move structure,  Soft robots: structure unconstrained, actuators impose forces/constraints to move structure
%Due to their soft and deformable structure, FREEs (and fluid driven soft actuators in general) differ in a fundamental way from traditional actuators.
%An electric motor, for example, essentially combines a kinematic constraint (the rotation axis of the motor, which is physically defined by a pair of bearings) with a force generating element (the electromagnetic forces, which create the motor torque).
%Since the motion of such an actuator is inherently limited to one dimension, multiple actuator stages are typically combined in \emph{series} to achieve multi degree of freedom (DOF) motions. 
%This is the prevalent design, for example, in industrial robotic arms.
%In contrast, in a soft actuator, the force generating element is not supported by any physical kinematic constraints.
%The actuator produces a spatial force without constraining the motion to happen exclusively in the direction of this force.
%Because of this, soft actuators are particularly well suited to be combined in \emph{parallel}, where the forces of the individual actuators are superimposed to generate a multi-dimensional spatial force.
%Such parallel combinations enable particularly compact designs of multi-DOF motion stages.
%They have equivalents in the world of traditional robotic systems, such as the well-known Steward Platform.
%However, in such systems, the complexity is much higher, as each individual actuator needs to be combined with five additional joints to overcome the inherent kinematic constraints.


%% Outline of paper
%This paper explores the potential of combining different types of FREE actuators in parallel to achieve fully controllable spatial forces.
%This work thus expands on the existing literature regarding fiber-reinforced fluid-driven actuators, which has focused mainly on the kinematics  \cite{bishop2015design, connolly2015mechanical, felt2018closed, krishnan2015kinematics} or kinetics \cite{bishop2013force, bruder2017model, sedal2017constitutive} of individual actuators, or the kinematics of parallel combinations of actuators \cite{bishop2012parallel, bishop2012parallelsynth}.
%Here, we study which combinations and configurations of FREEs enable effective control of multi-DOF forces.
%To this end, we present a novel way to represent and calculate actuator forces in terms of a state dependent fluid-Jacobian.
%This concept readily extends from a single soft actuator to parallel combinations of actuators.
%Our design and modelling methodology is employed and evaluated experimentally on a two degree of freedom test bench.



% %% Existing literature 
% Existing literature regarding fiber-reinforced fluid-driven actuators has focused on modeling the kinematics  \cite{bishop2015design, connolly2015mechanical, felt2018closed, krishnan2015kinematics} and kinetics \cite{bishop2013force, bruder2017model, sedal2017constitutive} of individual actuators, or the kinematics \cite{bishop2012parallel, bishop2012parallelsynth} and kinetics cite(BRUDER IROS) of parallel combinations of actuators . The aim of this work is to extend this work to control robotic systems that are actuated by parallel combinations of FREEs.



