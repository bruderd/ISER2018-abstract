\section{Technical Approach}    \label{sec:technical-approach}
% Technical Approach

%% We control end effector by controlling actuator pressures
A parallel combination consists of several actuators arranged such that one end is attached to a common ground while the other is attached to a common end effector. We desire to control the position of that end effector by varying the internal pressures inside of the actuators. For a desired end effector position, we use a force balance model to solve for a corresponding set of actuator pressures.

%% End effector forces can be written as a function of pressure and state
With the internal pressure and displacements of the actuators described by state vectors $\p$ and $\q$, respectively, the position and orientation of the end effector described by a state vector $\x$, and assuming that an inverse kinematics function $\q(\x)$ allows the computation of  $\q$ in terms of $\x$ , the end effector force $\f$ can be expressed in terms of $\x$ and $\p$ as 
\begin{align}
    \f (\x, \p) &= \J_x^T (\x) \p + \f_{\tx{elast}} \left( \q(\x) \right),
\end{align}
where $\J_x = \frac{\partial \vec{V}}{\partial \x}$ is the \emph{fluid Jacobian} which relates the change in volume of the actuators $\V$ to the change in the state of the end effector $\x$ (cite: IROS).

%% Solving for equilibrium points
This expression enables us to determine the pressure required to actuate the system toward a desired equilibrium state, $\x_\tx{des}$, by solving the following equation for $\p$,
\begin{align}
    0 &= \f (\x_{\tx{des}}, \p) + \f_\tx{load} =  \J_x^T (\x_\tx{des}) \p + \f_{\tx{elast}} \left( \q(\x_\tx{des}) \right) + \f_\tx{load} , 
    \label{eq:pequation}
\end{align}
where $\f_\tx{load}$ is the force imposed by external loads.
Since \eqref{eq:pequation} is a linear equation in $\p$, it is amenable to efficient numerical solving methods. In principle, \eqref{eq:pequation} may have multiple solutions, so we restructure it as a quadratic program that minimizes the magnitude of the pressure input with inequality constraints ensure that the minimizer solves \eqref{eq:pequation} to within a desired tolerance, tol.

\begin{equation}
\begin{aligned}
    & \underset{\p}{\text{minimize}}
    & & \p^T Q \p \\
    & \text{subject to}
    & & \mtx{\J_x^T \\ 
            -\J_x^T \\
            -\text{I} \\ \text{I}} \p
        \leq 
        \mtx{- \left( \f_{\tx{elast}} + \f_\tx{load} \right) + \text{tol} \\
            \f_{\tx{elast}} + \f_\tx{load} + \text{tol} \\
            -\p^\text{min} \\ \p^\text{max}}
\end{aligned}
\end{equation}

This quadratic program can be performed quickly enough to enable equilibrium position control of the end effector by iterative solving \eqref{eq:pequation} in real time.
















































%% TEXT FROM IROS PAPER FOR REFERENCE %%%%%%%%%%%%%%%%%%%%%%%%%%%%%%%%%%%%%%%%%%%%%%%%%%%%%%%%%%%%%

%\subsection{Model of a Parallel Combination of FREEs}
%The net force generated by parallel combinations of FREEs at the end effector (mostly following conventions of the IROS paper)
%\begin{align}
%    \f(\q, \p) &= \sum_{i=1}^n \f_{i} = \sum_{i=1}^n \D_i \T_i \\
%    &= \sum_{i=1}^n \D_i \left( \J^T_{q, i} (\q_i) p_i + \C_{q, i} \q_i \right) \\
%    &= \sum_{i=1}^n \bar{J}^T_{x, i} (\vec{x}) p_i + \C_{x, i} \q_i(\x),
%\end{align}
%where $\J_{x, i} = \J_{q, i} \D_i^T$ is the fluid Jacobian of the $i$th FREE expressed in end effector coordinates and $\C_x = \D_i \C_{q, i}$ is the stiffness contribution of the the $i$th FREE on the stiffness of the end effector. 
%This can be written compactly in matrix notation as
%\begin{align}
%    \f (\x, \p) &= \J_x^T (\x) \p + \C_x \q(\x)
%\end{align}
%with the overall fluid Jacobian $\J_x$ and overall stiffness $\C_x$
%\begin{align}
%    \J_x &= \mtx{\J^T_{x, 1} & \J^T_{x, 2} & \cdots & \J^T_{x, n}}^T \\
%    \C_x &= \mtx{\C_{x,1} & \C_{x,2} & \cdots & \C_{x,n}}
%\end{align}
